\documentclass[12pt, a4paper]{article}
\usepackage{xcolor}
\usepackage{amsmath}
\usepackage{amsfonts}
\usepackage{ragged2e}
\usepackage{amssymb}
\usepackage[T1]{fontenc}
\usepackage[utf8]{inputenc}
\usepackage[italian]{babel}
\usepackage{blindtext}



\title{Riassunto Comand SQL}
\author{erPaffo}

\renewcommand{\baselinestretch}{0.2}

\setlength{\hoffset}{-18pt}         
\setlength{\oddsidemargin}{0pt} % Marge gauche sur pages impaires
\setlength{\evensidemargin}{0pt} % Marge gauche sur pages paires
\setlength{\marginparwidth}{54pt} % Largeur de note dans la marge
\setlength{\textwidth}{481pt} % Largeur de la zone de texte (17cm)
\setlength{\voffset}{-18pt} % Bon pour DOS
\setlength{\marginparsep}{7pt} % Séparation de la marge
\setlength{\topmargin}{0pt} % Pas de marge en haut
\setlength{\headheight}{13pt} % Haut de page
\setlength{\headsep}{4pt} % Entre le haut de page et le texte
\setlength{\footskip}{27pt} % Bas de page + séparation
\setlength{\textheight}{720pt} % Hauteur de la zone de texte (25cm)

\usepackage{natbib}
\usepackage{graphicx}


\begin{document}
	\maketitle
	\tableofcontents
	\begin{frame}
		
		\section{SELECT}
		
			L'istruzione \textbf{select} seleziona gli attributi su cui si fonderà la query.  \\
			 %\begin{center}
		
			\textcolor{red}{\textbf{SELECT}} \emph{Attributo, ..., Attributo}  \\
 			\textcolor{red}{\textbf{FROM}} \emph{Tabella}  \\
			\textcolor{red}{\textbf{WHERE}} \emph{Condizione}  \\
			%\end{center} 
		
		\section{AS}
			\begin{itemize} 
			\item Per ridenominare un titolo di un attributo usiamo \textbf{as} nella clausola \textbf{select} 
			\item Per assegnare un nuovo \textbf{alias} alle tabelle usiamo \textbf{as} nella clausola \textbf{from}
			\end{itemize}

			\begin{center}
			\justifying
			\textbf{SELECT} \textbf{p}.nome \textbf{as} name,  \textbf{p}.reddito \textbf{as} salary \\
 			\textbf{FROM} persone \textbf{as} p  \\
			\textbf{WHERE} p.eta < 30  \\
			\end{center} 

		\section{LIKE}
			L'operatore \textbf{like} permette di verificare se la stringa appartiene a una espressione regolare definita dopo \textbf{\_}.
			\begin{center}
			\justifying
			\textbf{SELECT} \textbf{*} \\
 			\textbf{FROM} persone \\
			\textbf{WHERE} nome \textbf{like} \textbf{'A\_d\%'}  \\
			\end{center} 
			
		\section{JOIN}
			\begin{itemize} 
		
			\item Il \textbf{self-join, o join implicito,} viene fatto: 
				\begin{center}
				\justifying
				\textbf{SELECT} p1.nome, p1.eta, p1.reddito, p2.nome,  p2.eta \\
	 			\textbf{FROM} persone p1,  persone p2 \\
				\textbf{WHERE} p1.reddito = p2.reddito and p1.nome < p2.nome \\
				\end{center} 
				
			\item Il \textbf{join esplicito} viene fatto: 
				\begin{center}
				\justifying
				\textbf{SELECT} p.nome \\
	 			\textbf{FROM} persone p \textbf{join} citta c \textbf{on} p.cittanascita = c.nome \\
				\textbf{WHERE} p.eta > 18 \\
				\end{center} 
				
			\item Per fare il \textbf{prodotto cartesiano} utilizziamo \textbf{cross join}
				\begin{center}
				\justifying
				\textbf{SELECT} *\\
	 			\textbf{FROM} persone p \textbf{cross join} citta c 
				\end{center} 
			\item Per fare il \textbf{join esterno} utilizziamo \textbf{left outer join}:
				\begin{center}
				\justifying
				\textbf{SELECT} paternita.figlio, padre, madre\\
	 			\textbf{FROM} paternita \textbf{left join} maternita \textbf{on} paternita.figlio = maternita.figlio
				\end{center} 

			\end{itemize}	
		
		\section{ORDER BY}
			\textbf{Order by} serve per definire l'ordine in una query su un argomento
				\begin{center}
				\justifying
				\textbf{SELECT} nome, reddito\\
	 			\textbf{FROM} persone  \\
				\textbf{WHERE} eta < 30 \\
				\textbf{ORDER BY} nome \textbf{(desc)}, reddito
				\end{center} 
		
		\section{LIMIT}
			\textbf{Limit} serve per definire il numero massimo di elimenti in una query su ordinata
				\begin{center}
				\justifying
				\textbf{SELECT} nome,  reddito\\
	 			\textbf{FROM} persone \\
				\textbf{WHERE} eta < 30 \\
				\textbf{ORDER BY} nome \textbf{(desc)} \\
				\textbf{LIMIT} 2
				\end{center} 

		\section{Operatori Aggregati}
			\subsection{COUNT}
				\textbf{Count} calcola il minimo di un certo attributo
					\begin{center}
					\justifying
					\textbf{SELECT} \textbf{count(*)} \\
		 			\textbf{FROM} persone \\
					\end{center} 
			
			\subsection{MIN}
				\textbf{Min}
					 calcola il minimo di un certo attributo
					\begin{center}
					\justifying
					\textbf{SELECT} \textbf{min(*)} \\	
		 			\textbf{FROM} persone \\
					\end{center} 
	
			\subsection{MAX}
				\textbf{Max} 
					calcola il massimo di un certo attributo
					\begin{center}
					\justifying
					\textbf{SELECT} nome,  \textbf{max(reddito)} \\
		 			\textbf{FROM} persone \\
					\end{center} 
	
			\subsection{AVG}
				\textbf{Avg} 
					calcola la media di un certo attributo
					\begin{center}
					\justifying
					\textbf{SELECT} \textbf{avg(reddito)} \\
		 			\textbf{FROM} persone \\
					\end{center} 

			\subsection{SUM}
				\textbf{Sum} calcola la somma di un certo attributo
					\begin{center}
					\justifying
					\textbf{SELECT} \textbf{sum(*)} \\
		 			\textbf{FROM} persone \\
					\end{center} 

		

		\section{DISTINCT}
			\textbf{Distinct} permette di trovare i valori distinti di un attributo. \\
Può essere utilizzato anche nelle operazioni aggreagate
				\begin{center}
				\justifying
				\textbf{SELECT} distinct cognome, filiale\\
	 			\textbf{FROM} impiegati \\
				\\oppure \\ \\
				\textbf{SELECT} count(distinct reddito)\\
	 			\textbf{FROM} persone \\
				\end{center} 

		\section{GROUP BY}
			\textbf{Group by} raggruppa le tuple che hanno lo stesso valore sugli attributi chiamati
			\begin{center}
				\justifying
				\textbf{SELECT} distinct cognome, filiale\\
	 			\textbf{FROM} impiegati \\
				\textbf{GROUP BY} padre 
			\end{center}

			\subsection{HAVING}
				\textbf{Having} indica la condizione sui gruppi
				\begin{center}
					\justifying
					\textbf{SELECT} padre,  avg(f.reddito) \\
		 			\textbf{FROM} persone f join paternita on figlio = f.nome \\
					\textbf{GROUP BY} padre \\
					\textbf{HAVING} avg(f.reddito) > 25
				\end{center}

	

	\section*{Sintesi}
		\begin{center}
		\justifying
		\textcolor{blue}{\textbf{select}} \emph{Attributo, ..., Attributo}  \\
 		\textcolor{blue}{\textbf{from}} \emph{ListaTabell}  \\
		\textcolor{blue}{\textbf{where}} \emph{CondizioniSemplici} \text{]}  \\
		\text{[} \textcolor{blue}{\textbf{group by}} \emph{ListaAttributiDiRaggruppamento} \text{]}   \\
		\text{[} \textcolor{blue}{\textbf{having}} \emph{CondizioniAggregate} \text{]}  \\
		\text{[} \textcolor{blue}{\textbf{order by}} \emph{ListaAttributiDiOrdinamento} \text{]}  \\
		\text{[} \textcolor{blue}{\textbf{limit}} \emph{numer} \text{]}  \\
		\end{center}
	
	\end{frame}

		\section{}
			\textbf{}
			\quad \textbf{select}
			\quad \textbf{from]
			\quad \textbf{where}
	


\end{document}