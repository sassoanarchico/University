\documentclass[a4paper]{book}
\usepackage{xcolor}
\usepackage{amsmath}
\usepackage{amsfonts}
\usepackage{ragged2e}
\usepackage{blindtext}
\title{Riassunto Comand SQL}
\author{erPaffo}
\begin{document}
	\maketitle
	\tableofcontents
	\begin{frame}
		
		\section{SELECT}
		
			L'istruzione \textbf{select} seleziona gli attributi su cui si fonderà la query.  \\
			\begin{center}
			\justifying
			\textcolor{red}{\textbf{select}} \emph{Attributo, ..., Attributo}  \\
 			\textcolor{red}{\textbf{from}} \emph{Tabella}  \\
			\textcolor{red}{\textbf{where}} \emph{Condizione}  \\
			\end{center} 
		
		\section{AS}
			Per ridenominare usiamo \textbf{as} nella clausola \textbf{from}
			\begin{center}
			\justifying
			\textcolor{red}{\textbf{select}}p.nome \textbf{as} name, p.reddito as salary  \\
 			\textcolor{red}{\textbf{from}} \emph{Tabella}  \\
			\textcolor{red}{\textbf{where}} \emph{Condizione}  \\
			\end{center} 

	\end{frame}




\end{document}