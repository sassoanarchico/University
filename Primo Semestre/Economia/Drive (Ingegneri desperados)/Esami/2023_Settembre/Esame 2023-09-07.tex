\documentclass{article}
\usepackage{amsmath}
\usepackage{amsfonts}
\usepackage[a4paper,width=140mm,top=15mm,bottom=15mm]{geometry}
\usepackage{hyperref}
\usepackage{multicol}
\usepackage{float}
\usepackage{graphicx} % Pacchetto per la gestione delle figure
\usepackage{eurosym} % Pacchetto per il simbolo dell'euro
\hypersetup{
    colorlinks,
    citecolor=black,
    filecolor=black,
    linkcolor=black,
    urlcolor=black
}

\DeclareUnicodeCharacter{2212}{-}




\title{Esame 07/09/2023}
\author{Mic Leig}
\begin{document}
\maketitle



\section{Esercizio {1}}
I numeri non sono quelli dell'esercizio, li ho messi a caso a memoria.
$i = 10\%$

Niente qua ci stava un'immagine ma è persa

Calcolare progetto migliore usando \textbf{Indice di Profittabilità}


\section{Esercizio 2}
Con $i = 6\%, f = 3\%, t=0$ calcolare valore attuale, montante corrente di 8 pagamenti uguali da 1500 \euro
costanti, e anche F( vabbe la somma prima e dopo non ricordo mo come si dice ma è il solito esercizio di questo tipo)

\section{Esercizio 3}
Eguaglianza:

a) 10 pagamenti dal 6 al 15 anno da 3000 \euro;

b) trew pagamenti al 6,9,11 anno.

$i = 4\%$.

\section{Domande di Teoria}
1) Descrivere le decisioni di investimento di un'impresa
2) Descrizione e finanziamento di una startup
3) Principi base della contabilità
4) Criteri con vantaggi e svantaggi del budget

















\end{document}
